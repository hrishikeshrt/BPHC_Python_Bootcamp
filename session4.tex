\documentclass[12pt]{article}
\usepackage{geometry}
\usepackage{listings}
\usepackage[T1]{fontenc}
\usepackage[utf8]{inputenc}
\usepackage{listingsutf8}
\usepackage{xcolor}
\usepackage{amssymb}
\usepackage{hyperref}
\usepackage{graphicx}
\usepackage{fancyhdr}
\usepackage{amsmath}
\usepackage{FiraMono}
\usepackage{tikz}
\usepackage{transparent}
\usepackage[firstpage=false]{draftwatermark}

\usepackage[sfdefault,light]{FiraSans}
\renewcommand{\familydefault}{\sfdefault}  % Set sans-serif font as the default
\newcommand{\md}{\firamedium}
\renewcommand{\textmd}[1]{{\firamedium #1}}

\geometry{a4paper, margin=1in}
\pagestyle{fancy}
\fancyhf{}
\rhead{Python Bootcamp - Session 4}
\lhead{Python Integration with Other Technologies}

% Python style for listings
\definecolor{codegreen}{rgb}{0,0.6,0}
\definecolor{codegray}{rgb}{0.5,0.5,0.5}
\definecolor{codepurple}{rgb}{0.58,0,0.82}
\definecolor{backcolour}{rgb}{0.96,0.96,0.94}

\lstdefinestyle{pythonstyle}{
    backgroundcolor=\color{backcolour},
    commentstyle=\color{codegreen},
    keywordstyle=\color{magenta},
    numberstyle=\tiny\color{codegray},
    stringstyle=\color{codepurple},
    basicstyle=\linespread{0.95}\selectfont\ttfamily\scriptsize,
    breakatwhitespace=false,
    breaklines=true,
    captionpos=b,
    keepspaces=true,
    columns=fullflexible,
    numbers=left,
    numbersep=5pt,
    showspaces=false,
    showstringspaces=false,
    showtabs=false,
    tabsize=2,
    frame=single,
    framesep=6pt,
    rulecolor=\color{black!10}
}

\lstset{style=pythonstyle,inputencoding=utf8}
\setlength{\fboxsep}{0pt}
\setlength{\fboxrule}{0pt}

\SetWatermarkText{\includegraphics[width=0.5\paperwidth]{assets/bits_pilani_logo_faint.png}}
\SetWatermarkScale{1}
% \SetWatermarkLightness{1}
\SetWatermarkAngle{0}

\begin{document}

\title{{\Large \textmd{Python Bootcamp - Session 4}}\\Python Integration with Other Technologies}
\author{Hrishikesh Terdalkar}
\date{November 30, 2025}

\maketitle

\section*{Session Overview}
Session 4 builds on the shared datasets from the previous session and demonstrates how to integrate them with databases, external data sources, parallel compute, and a lightweight dashboard. The goal is to provide a complete reference workflow that you can adapt to your own research problems.

\section*{What You Will Build}
\begin{itemize}
    \item A SQLite-backed store of the Session 3 datasets using SQLAlchemy models.
    \item A small API collector that fetches (or mocks) external context and analyses it.
    \item A parallel Monte Carlo benchmark with a side-by-side sequential comparison.
    \item A minimal Flask dashboard that exposes analysis results as HTML and JSON.
    \item An integration script that ties all pieces together and writes a report.
\end{itemize}

\section*{Session Plan}
\begin{itemize}
    \item Database ingest and quick statistics.
    \item API integration (mock weather/material properties) and CSV export.
    \item Parallel processing demo and speedup discussion.
    \item Dashboard walkthrough (CLI mode first, optional Flask run).
    \item Integration demo and report generation.
\end{itemize}

\section*{Skills You'll Practice}
\begin{itemize}
    \item Persisting experiment data with SQLAlchemy models (SQLite).
    \item Designing small API clients, handling JSON, and saving tidy CSVs.
    \item Running CPU-bound work in parallel and interpreting speedups.
    \item Exposing analysis via a minimal Flask dashboard and JSON routes.
    \item Integrating outputs across tools and generating a final report.
\end{itemize}

\section*{Learning Outcomes}
By the end of this session, you should be able to:
\begin{itemize}
    \item Create and query a SQLite database using SQLAlchemy ORM classes.
    \item Collect or mock external API data and analyze simple trends.
    \item Compare sequential vs parallel runs and reason about efficiency.
    \item Serve analyses over HTTP (Flask) and export artefacts on demand.
    \item Produce an end-to-end integration report from combined components.
\end{itemize}
\clearpage
\section*{Theory Essentials}
\begin{itemize}
    \item \textbf{ORM basics}: SQLAlchemy maps Python classes to tables so you can query without handwritten SQL.
    \item \textbf{API fundamentals}: Keep authentication and headers centralised; persist raw responses before transforming.
    \item \textbf{Parallelism}: Use processes for CPU-bound workloads; threads help when the bottleneck is I/O.
    \item \textbf{Amdahl's law}: Speedup is limited by the serial fraction; profile before parallelising.
    \item \textbf{Web architecture}: Separate routes (HTTP) from analysis functions so you can test logic without a server.
\end{itemize}

\section*{Prerequisites and Setup}
Use the same environment from Session 3 or create a fresh one. The examples assume you have already generated the shared dataset with \texttt{make data}.
\begin{lstlisting}[language=bash]
# Install dependencies into your active environment
pip install -r requirements.txt

# Prepare shared data produced in Session 3
make data
\end{lstlisting}

\section*{How to Run the Examples}
\begin{enumerate}
    \item Persist CSV data to SQLite:
    \begin{lstlisting}[language=bash]
python session4/01_database.py
# Creates research_data.db and writes statistics + CSV export
    \end{lstlisting}
    \item Collect and analyze API data (mocked for offline use):
    \begin{lstlisting}[language=bash]
python session4/02_api_integration.py
# Produces weather_analysis_data.csv for later use
    \end{lstlisting}
    \item Compare sequential vs parallel workloads:
    \begin{lstlisting}[language=bash]
python session4/03_parallel_processing.py
# Observe speedup and verify identical results across modes
    \end{lstlisting}
    \item Explore the dashboard (demo or server):
    \begin{lstlisting}[language=bash]
python session4/04_research_dashboard.py
# or run a local server
flask -{}-app session4/04_research_dashboard.py run -{}-port 5000
    \end{lstlisting}
    \item Tie everything together and generate a report:
    \begin{lstlisting}[language=bash]
python session4/05_integration_demo.py
# Writes integrated_system_report.json
    \end{lstlisting}
\end{enumerate}

\section*{Lab Guide}
\subsection*{01\_database.py}
\begin{itemize}
    \item Defines SQLAlchemy models and a helper class to ingest DataFrames.
    \item Adds data in batch and computes basic per-parameter statistics.
    \item Exports an experiment to CSV for downstream tools.
\end{itemize}

\subsection*{02\_api\_integration.py}
\begin{itemize}
    \item Demonstrates a simple API client pattern with a session and headers.
    \item Uses the free Open-Meteo weather API when the network is available, and falls back to deterministic mock data otherwise.
    \item Produces a tidy CSV for reuse in plotting or dashboards.
\end{itemize}

\subsection*{03\_parallel\_processing.py}
\begin{itemize}
    \item Benchmarks sequential vs parallel Monte Carlo simulations.
    \item Shows chunked batch processing with and without processes.
    \item Emphasises correctness: compare results across modes before celebrating speed.
\end{itemize}

\subsection*{04\_research\_dashboard.py}
\begin{itemize}
    \item Minimal Flask app exposing analysis as HTML/JSON.
    \item Generates plots in-memory and returns base64-encoded PNGs.
    \item Includes endpoints to analyze, plot, export, and list experiments.
\end{itemize}

\subsection*{05\_integration\_demo.py}
\begin{itemize}
    \item Generates a small dataset, writes it to CSV and SQLite, and combines it with mock API data.
    \item Runs a few parallel jobs, creates plots, and writes a concise JSON integration report.
\end{itemize}

\section*{Best Practices}

\subsection{Error Handling}
\begin{itemize}
    \item Database: check connections, wrap commits in \texttt{try/except}, and roll back on failure.
    \item APIs: set timeouts, catch \texttt{requests} exceptions, and log raw payloads before transforming.
    \item Parallel jobs: handle exceptions inside worker functions and return error details to the parent.
    \item Flask: validate inputs from \texttt{request.json} and return proper status codes.
\end{itemize}

\subsection{Documentation}
\begin{itemize}
    \item Document ORM models (field purpose/units) and any expected CSV schema.
    \item Record API endpoints, auth method, and rate limits near the client class.
    \item Include CLI examples (copy/paste) for each module; keep \texttt{--help} informative.
\end{itemize}

\section*{Common Pitfalls}
\begin{itemize}
    \item Activate your virtual environment before running the Flask CLI; otherwise \texttt{flask} may not find dependencies.
    \item If \texttt{engineering\_test\_data.csv} is missing, run \texttt{make data} to regenerate the shared dataset.
    \item Firewall prompts can block the Flask server from binding to a port; allow local connections if prompted.
\end{itemize}

\section*{More Examples}

\noindent Persist correlations after database ingest:
\begin{lstlisting}[language=Python]
df = db.get_experiment_data("thermal_study_001")
num = df.pivot_table(index="timestamp", columns="parameter", values="value")
num.corr().to_csv("correlations.csv")
\end{lstlisting}
\noindent Simple parallel parameter sweep skeleton:
\begin{lstlisting}[language=Python]
from concurrent.futures import ProcessPoolExecutor
def simulate(param):
    # compute something CPU-bound
    return param, param**2
with ProcessPoolExecutor() as ex:
    results = list(ex.map(simulate, range(8)))
\end{lstlisting}

\section*{Practice Exercises}
\begin{enumerate}
    \item Add a new SQLAlchemy model that stores computed rolling means, then write a short script to populate it.
    \item Replace the mock weather generator with a free API of your choice (for example, Open-Meteo), and save raw JSON before flattening to CSV.
    \item Extend the dashboard with one extra route that returns a correlation heatmap as a PNG.
\end{enumerate}

\section*{Further Reading}
\begin{itemize}
    \item Official Python documentation: \url{https://docs.python.org/3/}
    \item pandas documentation (data analysis): \url{https://pandas.pydata.org/docs/}
    \item SQLAlchemy documentation (databases): \url{https://docs.sqlalchemy.org/}
    \item Flask documentation (web apps): \url{https://flask.palletsprojects.com/}
    \item Open-Meteo API docs (example of a free, research-friendly API):\\\url{https://open-meteo.com/en/docs}
\end{itemize}

\clearpage
\section*{Appendix: Full Code Listings}

\subsection*{session4/01\_database.py}
\lstinputlisting[language=Python, caption={SQLAlchemy database integration}]{session4/01_database.py}

\subsection*{session4/02\_api\_integration.py}
\lstinputlisting[language=Python, caption={API integration (mock weather and materials)}]{session4/02_api_integration.py}

\subsection*{session4/03\_parallel\_processing.py}
\lstinputlisting[language=Python, caption={Parallel processing and Monte Carlo}]{session4/03_parallel_processing.py}

\subsection*{session4/04\_research\_dashboard.py}
\lstinputlisting[language=Python, caption={Research dashboard with Flask}]{session4/04_research_dashboard.py}

\subsection*{session4/05\_integration\_demo.py}
\lstinputlisting[language=Python, caption={Integration demo: DB + APIs + parallel}]{session4/05_integration_demo.py}

\end{document}
